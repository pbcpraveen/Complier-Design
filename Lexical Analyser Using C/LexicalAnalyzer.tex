% Created 2020-01-06 Mon 14:41
\documentclass[11pt]{article}
\usepackage[latin1]{inputenc}
\usepackage[T1]{fontenc}
\usepackage{fixltx2e}
\usepackage{graphicx}
\usepackage{longtable}
\usepackage{float}
\usepackage{wrapfig}
\usepackage{rotating}
\usepackage[normalem]{ulem}
\usepackage{amsmath}
\usepackage{textcomp}
\usepackage{marvosym}
\usepackage{wasysym}
\usepackage{amssymb}
\usepackage{hyperref}
\tolerance=1000
\usepackage{palatino}
\usepackage[top=1in,bottom=1.25in,left=1.25in,right=1.25in]{geometry}
\usepackage{setspace}
\setcounter{secnumdepth}{1}
\author{Praveen Kumar R}
\date{\today}
\title{Exercise 1: Lexical Analyser}
\hypersetup{
  pdfkeywords={},
  pdfsubject={},
  pdfcreator={Emacs 25.2.2 (Org mode 8.2.10)}}
\begin{document}

\maketitle
\begin{export}
\linespread{1.2}
\end{export}
\begin{center}
\begin{tabular}{lr}
Assignment & 1\\
Reg No & 312217104114\\
Name & Praveen Kumar R\\
\end{tabular}
\end{center}


\section{Program}
\label{sec-1}
\begin{verbatim}
#include<stdio.h>
#include<string.h>
#include <sys/types.h>
#include <sys/stat.h>
#include <fcntl.h>
#include <unistd.h>
#include <stdlib.h>
#include <ctype.h>

void main()
{
	FILE * fp;
	int count=0;
	char * line = NULL;
	size_t len = 0;
	ssize_t linelen;
	char store1[10][100];
	char store2[10][100];
	fp = fopen("./in.c", "r");

	while ((linelen = getline(&line, &len, fp)) != -1) {
		if(line[0] == '#'){
			for(int i=0;i<strlen(line);i++){
				if(line[i] != '\n') printf("%c",line[i]);
			}
        	printf(" - preprocessor directive\n");
		}
		char* int1 = strstr(line,"int ");
		char* for1 = strstr(line,"for(");
		char* if1 = strstr(line,"if(");
		char* else1 = strstr(line,"else");

		int declare = 0;
		int conditional = 0;

		if(int1 != NULL){
			declare = 1;
			printf("int - keyword\n");
			char* p = int1;
			 char str[2];
			 char*t=p;
			 int jumplen=strlen("int ");
			 t=t+4;
			 while(*t!='\0')
			   {
			     char c=*t;
			     str[0]=c;
			     str[1]='\0';
			     strcpy(store1[count],str);
			     t=t+1;
			     if(isalnum(*(t)))
			       {
				 char c=*t;
				 str[0]=c;
				 str[1]='\0';
				 strcat(store1[count],str);
				 t=t+1;
			       }
			     if(*t=='=')
			       {
				 t=t+1;
				 while(isdigit(*(t)))
				   {
				     char c=*t;
				     str[0]=c;
				     str[1]='\0';
				     strcat(store2[count],str);
				     t=t+1;
				   }
			       }
			     if((*(t))==','|*t==';')
			       {
				 count=count+1;
				 t=t+1;
			       }
			     
			   }
			}
		if(for1 != NULL)
			printf("for - keyword\n");
		if(if1 != NULL){
			printf("if - keyword\n");
			conditional = 1;
		}
		if(else1 != NULL)
			printf("else - keyword\n");

		char* templine;
		templine = line;

		int first = 1;
		if(declare == 1){
			while(templine != NULL){
				if(first == 1){
					templine = strstr(templine," ");
					first = 0;
				}
				else{
					printf(", - special character\n");
				}

				int equindex;
				for(int z=0;z<strlen(templine);z++){
					if(*(templine+z) == '='){
						equindex=z;
						break;
					}
				}

				for(int j=1;j<equindex; j++){
					printf("%c",*(templine+j));
				}

				printf(" - variable\n");

				printf("= - assignment operator\n");
				templine = strstr(templine, "=");

				int commaindex;
				for(int z=0;z<strlen(templine);z++){
					if(*(templine+z) == ','){
						commaindex=z;
						break;
					}
				}

				for(int j=1;j<commaindex; j++){
					printf("%c",*(templine+j));
				}
				printf("- constant\n");
				templine = strstr(templine, ",");
			}

		}

		char* main1 = strstr(line, "main(");
		char* printf1 = strstr(line, "printf(");


		if(main1 != NULL || printf1 != NULL){
			for(int i=0;i<strlen(line);i++){
				if(line[i]=='\t' || line[i]==';' || line[i] == '\n'){
					
				}
				else{
					printf("%c",line[i]);
				}
			}
			printf(" - function call\n");
		}
		char* popen = strstr(line, "{");
		if(popen != NULL) printf("{ - special character\n");

		char* semicolon = strstr(line,";");
		if(semicolon != NULL)printf("; - special character\n");

		char* pclose = strstr(line, "}");
		if(pclose != NULL) printf("} - special character\n");

		char* bracket_open = strstr(line, "(");
		if(bracket_open != NULL  && main1 == NULL && printf1 == NULL) 
                                      printf("(- special character\n");

		char* tempvar;
		if(conditional == 1){
			tempvar = strstr(line,"(");
			int i;

			int condition;
			for(int z=0;z<strlen(tempvar);z++){
				if(*(tempvar+z) == '<' || *(tempvar+z) == '>'){
					condition=z;
					break;
				}
			}
			for(int j=1;j<condition;j++){
				printf("%c",*(tempvar+j));
			}
			printf(" - variable\n");
			char* tempvar1 = strstr(tempvar,"<");
			char* tempvar2 = strstr(tempvar, ">");
			if(tempvar1!=NULL)tempvar = tempvar1;
			if(tempvar2!=NULL)tempvar = tempvar2;

			printf("%c - condition\n",*(tempvar));

			for(int z=1;z<strlen(tempvar);z++){
				if(*(tempvar+z) == ')'){
					condition=z;
					break;
				}
				else{
					printf("%c",*(tempvar+z));
				}
			}

			printf(" - variable\n");
		}


		char* bracket_close = strstr(line, ")");
		if(bracket_close != NULL && main1 == NULL && printf1 == NULL) printf(") - special character\n");		

    }

    printf("\n\n\nSYMBOL TABLE\n");
    int base = 1000;
    for(int i=0;i<count;i++)
      {
	 printf(" %d \t int \t %s \t %s \t %d\n",i+1,store1[i],store2[i],base);
         base+=2;
      }

    
    fclose(fp);
}
\end{verbatim}

\section{Output}
\label{sec-2}
\begin{verbatim}
#include<stdio.h> - preprocessor directive
main() - function call
{ - special character
int - keyword
a - variable
= - assignment operator
10- constant
, - special character
b - variable
= - assignment operator
20- constant
; - special character
if - keyword
(- special character
a - variable
> - condition
b - variable
) - special character
printf( a is greater ) - function call
; - special character
else - keyword
printf( b is greater ) - function call
; - special character
} - special character



SYMBOL TABLE
 1 	 int 	 a 	 10 	 1000
 2 	 int 	 b 	 20 	 1002
\end{verbatim}
% Emacs 25.2.2 (Org mode 8.2.10)
\end{document}
